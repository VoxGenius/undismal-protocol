
\begin{table}[htbp]
\centering
\caption{Structural Break Test Results}
\label{tab:structural_breaks}
\begin{tabular}{lccccc}
\toprule
Test Period & Chow Test & P-value & CUSUM & CUSUM-SQ & Parameter \\
& Statistic & & & & Stability \\
\midrule
1975:Q1 & 2.84 & 0.092 & Stable & Stable & 0.15 \\
1980:Q1 & 8.92 & 0.003*** & Unstable & Stable & 0.42 \\
1985:Q1 & 4.25 & 0.039** & Stable & Stable & 0.22 \\
1990:Q1 & 12.45 & 0.000*** & Unstable & Unstable & 0.68 \\
1995:Q1 & 6.78 & 0.009*** & Stable & Unstable & 0.35 \\
2000:Q1 & 3.15 & 0.076* & Stable & Stable & 0.18 \\
2005:Q1 & 2.95 & 0.086* & Stable & Stable & 0.16 \\
2010:Q1 & 7.82 & 0.005*** & Unstable & Stable & 0.45 \\
2015:Q1 & 1.95 & 0.162 & Stable & Stable & 0.08 \\
\midrule
Sup-F Test & 15.67 & 0.001*** & & & \\
Exp-F Test & 8.95 & 0.003*** & & & \\
Ave-F Test & 6.42 & 0.008*** & & & \\
\midrule
Most Likely Break: & 1991:Q2 & & & & \\
95\% Confidence Interval: & [1990:Q3, & & & & \\
& 1992:Q1] & & & & \\
\bottomrule
\end{tabular}
\begin{tablenotes}
\footnotesize
\item *, **, *** indicate significance at 10\%, 5\%, and 1\% levels. Chow tests use 15\% trimming. CUSUM and CUSUM-SQ tests use 5\% significance bands. Parameter stability measured as rolling standard deviation of coefficient estimates. Sup-F, Exp-F, and Ave-F are Bai-Perron multiple break tests.
\end{tablenotes}
\end{table}
